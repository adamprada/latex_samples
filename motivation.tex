% Sample LaTeX document by Adam Přáda
%******************************************************************************
% DOCUMENT AND FONT SET-UP
%******************************************************************************
% First specify the class of document and body font size
\documentclass[11pt]{article}

% Always type input in UTF-8, which guarantees you can write directly most symbols
\usepackage[utf8]{inputenc}
% Using the T1 for font encoding ensures better non-ASCII chracters (as opposed to old OT1)
\usepackage[T1]{fontenc}
% Next command loads the Latin Modern family of fonts. These look just like LaTeX default Computer Modern, but has more available characters and symbols
%\usepackage{lmodern}


% Another good combination of font is Palatino:
\usepackage{newpxtext}
% with either the default math font:
%\usepackage{newpxmath}
% or the Euler math font, which is very nice
\usepackage[euler-digits,euler-hat-accent]{eulervm}

% A font designed by Knuth, the author of TeX
%\usepackage{concrete}
%\usepackage[euler-digits,euler-hat-accent]{eulervm} % Mean to go together with the Euler font


% A generally useful package, e.g. if you want to change font colour
\usepackage{xcolor}

% If writing a document in a different language than English, load the babel package
% Automatically translates words like Chaper, Figure, Table, etc.
%\usepackage[czech]{babel}
%******************************************************************************
% PAGE SET-UP
%******************************************************************************
% Choose the right paper format and margins
%\usepackage[a4paper, top=25mm,bottom=25mm, inner=4cm,outer=3cm]{geometry}
% Each of these can also be set in separate commands
\usepackage{geometry}
\geometry{a4paper}
%\geometry{margin=2.5cm}

% If you want to use multiple columns
%\usepackage{multicol}

% For nicer page headers
%\usepackage{fancyhdr} % Load the package
%\pagestyle{fancy} % Make pages fancy
%\fancyhead[LE,RO]{}
%\fancyhead[RE,LO]{\leftmark}
%\fancyhead[RE,LO]{Section \thesection}
%\fancyfoot[CE,CO]{\leftmark}
%\fancyfoot[LE,RO]{\thepage}
% E: Even page
% O: Odd page
% L: Left field
% C: Center field
% R: Right field
% H: Header
% F: Footer
%******************************************************************************
% PARAGRAPHS
%******************************************************************************
% If you do not want your paragraphs to be indented and you want a gap instead
\setlength{\parindent}{0pt}
\setlength{\parskip}{0.5em}

% If you want to use left or right aligned text, then it may be better to use
% this package,
\usepackage{ragged2e}
% that defines new capitalised commands:
% \Centering
% \RaggedLeft
% \RaggedRight
% These generally work better than their lowercase LaTeX counterparts 
%******************************************************************************
% FIGURES
%******************************************************************************
% For including images/figures
\usepackage{graphicx}
% If most images are in a different folder to the document
\graphicspath{ {figures/} }
%******************************************************************************
% MATHEMATICS
%******************************************************************************
% Basically a necessary package for typing maths
\usepackage{amssymb}
% For more mathematical symbols
\usepackage{amsmath}
% For bold symbols in mathematics
% => use \bm{} instead of \mathbf or \boldsymbol
% => use \hm instead of \heavysymbol
\usepackage{bm}
% Sometimes it may be useful to load the physics package
%\usepackage{physics}
%******************************************************************************
% LINKS IN THE DOCUMENT
%******************************************************************************
% By default I prefer to have links hidden
\usepackage[hidelinks]{hyperref}
% but you can play with them as well
%\usepackage{hyperref}
%\hypersetup{colorlinks=true,linkcolor=black,filecolor=magenta,urlcolor=blue}
%\urlstyle{same}
%******************************************************************************
% BIBTEX BIBLIOGRAPHY
%******************************************************************************
% I prefer to use the Biber backend
%\usepackage[backend=biber,style=chem-rsc,]{biblatex}
%\addbibresource{bibtex/PhD-first-year_report.bib} %Imports bibliography file
%******************************************************************************
% FOOTNOTES WITH SYMBOLS
%******************************************************************************
% I prefer to have footnotes labelled with symbols, to avoid confusion with citations
\renewcommand{\thefootnote}{\fnsymbol{footnote}}
%******************************************************************************
% NON-BREAKING HYPHENS
%******************************************************************************
% Always load as the last package
\usepackage[shortcuts]{extdash}
%Standard LaTeX dashes
%- Standard LaTeX hyphen
%-- Standard LaTeX en-dash
%--- Standard LaTeX em-dash

%extdash breakable dashes
%Words hyphened with these dashes can also be broken at other positions than the dash
%\-/ hyphen
%\-- en-dash
%\--- em-dash

%extdash unbreakable dashes
%No line breaks possible at the hyphen
%\=/ hyphen
%\== en-dash
%\=== em-dash
%******************************************************************************
% DOCUMENT BODY
%******************************************************************************
% If you want to use the default document title, then first set the variables
\title{Why \LaTeX?}
\author{Adam Přáda}
\date{25 July 2019}
% and then start the document and make the title
\begin{document}
\maketitle
Most people use Microsoft Word for typesetting, so the first question is:
\section{Why not Microsoft Word?}
There is plenty of reasons to use Word so let us focus on why not.
\subsection{The quality of the outcome}
Yes, it is possible to create professionally looking documents in Word, but one has to be extremely careful to make use of many of its functions, some of which you have probably never heard of. In LaTeX, the default looks great and if you intentionally make changes, then definitely for the better!
\subsection{Working with complicated documents}
If your document has any of the following: sections, chapters, figures, tables, footnotes, citations and references, a table of contents, appendices. etc.~, then LaTeX is going to make your life much easier by automatising much of the work.
\subsection{Working with large documents}
Microsoft Word can be a demanding program to run on a computer and even more so if the document is very large. LaTeX files are simple text files, that can handle any size and also allow you to split the document into smaller chunks.
\pagebreak
\subsection{Platform dependence/Portability}
LaTeX is available on Windows, Mac and pretty much all Linux distributions as well as online by Overleaf. Also, with LaTeX you basically cannot run into a situation, that you send someone a file and it does not work properly because of the version of the software (which one cannot say about Word).
\subsection{Money}
Microsoft Office is an expensive software package. LaTeX is an open-source typesetting system, which is free and there are many free graphical editors for all platforms as well.
\subsection{Openness}
Both .doc and .docx are closed formats created by Microsoft and not fully disclosed to general public. This means that no one except for Microsoft can make programs that reliably work with them. Sending someone a Word document is basically saying: ``You either buy this expensive piece of software that only one company in the world sells, or use a cheaper alternative, but face the consequences of not-full compatibility." With LaTeX there is plenty of editors and if you really wanted to, you could make your own one. You could also make your own improved version of LaTeX, if you feel like it. (And some people do.)

The least you can do in this regard is to begin using open document formats .odt, .ods, .odp, etc., which Microsoft Office can use, but LibreOffice, OpenOffice and many other programs can use as well. Because these are open, the compatibility will be much better and again, you could make your own program if you wanted to.
\subsection{Ethics}
Some people consider the practices that Microsoft uses to get to schools and state institutions unethical.\footnote[2]{Giving their software to schools and state institutions for free or a very low price means, that children will learn only with their software and state institutions will use their document formats to communicate with private entities. Individuals and companies are then basically forced to play the game, which closes the vicious circle.} This and the closed formats may make Microsoft ethically unacceptable for you, in which case you will seek an alternative.

\section{What is \LaTeX?}
Now that you hopefully want to give \LaTeX a try, we should take a look at what it actually is. Before \LaTeX, there was only \TeX. TeX\footnote[3]{The name consists of three capital Greek letters: T~=~$\tau$ (tau), E~=~$\varepsilon$ (epsilon), X~=~$\chi$ (chi), which happen to look just latin letters `t', `e' and `x'. The correct pronunciation would then be /t$\varepsilon$x/, where X is pronounced like \emph{ch} in \emph{Loch Ness}. However, in commonly used English words of Greek origin, `ch' that originates from $\chi$ is pronounced like `k' (e.g.~in architecture), so pronunciation /t$\varepsilon$k/ is also accepted.} is a typesetting system that was created by Donald Knuth, when he was displeased with how his book about programming was typeset. Because it was very demanding to use, Leslie Lamport created an extension of \TeX which became known as Lamport TeX or \LaTeX. In LaTeX the overly complicated typesetting language is hidden beneath easy-to-use commands, which suffice for vast majority of people. And the rest can always dig deeper and play with TeX itself.

\end{document}